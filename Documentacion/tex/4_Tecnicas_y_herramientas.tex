\capitulo{4}{Técnicas y herramientas}


En esta sección se explica brevemente la metodología utilizada en el desarrollo del trabajo de fin de grado. También veremos las herramientas elegidas para el llevar a cabo el mismo y su elección frente a otras.

\section{Técnicas}

\subsection{Scrum} 
Es aquel proceso que se aplica un conjunto de prácticas para trabajar en equipo y de esta manera obtener el mejor resultado posible en un proyecto.

Se realizan entregas parciales hasta realizar la entrega del producto final. Estas entregas parciales las denominamos Sprints, al final de cada Srpint el equipo se reúne y decide cuales son los objetivos de cara al próximo Sprint.

Scrum es indicado para aquellos proyectos en los que  tenemos requisitos poco definidos, que están abiertos cambios, ya que lo indicado entonces es realizar objetivos a corto plazo y así  agilizar el desarrollo del mismo.

\subsection{Web scraping}

Técnica utilizada para simular  la navegación de un humano en Internet con el fin de extraer información del sitio web. Los datos pueden ser almacenados y analizados en una base de datos central o en otros lugares de almacenamiento.  Son diversas las técnicas que se pueden utilizar, en concreto hemos realizado a través de peticiones al protocolo HTTP.

\section{Herramientas}

\subsection{Trello}

Trello (https://trello.com/) es una herramienta colaborativa que nos permite organizar nuestro  proyecto en tablones. En el tablero ponemos diferentes tarjetas y en cada una de ellas una lista.
Hemos decidido usarla principalmente por su sencillez y facilidad de uso. Otra razón que nos ha llevado a su uso es la aplicación para Android que nos permite estar  conectados en cualquier momento.

\subsection{GitHub}
Git Hub (https://github.com/) es una plataforma que nos permite el desarrollo colaborativo del software, alojando el repositorio de código en ella.
Su elección se debe al conocimiento de su funcionamiento, además de su buena adaptación a Ubuntu, sistema operativo en el que se ha trabajado y que mendiante sencillos comandos nos permite utilizarla fácilmente.

\subsection{Oracle VM Virtual Box}
Virtual Box (https://www.virtualbox.org/) es un software libre con el cual hemos trabajado a lo largo de toda la carrera, cumple bien nuestras necesidades además de ser sencillo en su uso. Hay algunas alternativas como por ejemplo Hyper X que se pueden instalar en Windows, pero es necesario tener el sistema operativo en una versión Pro a la cual no tenemos acceso.

\subsection{Servidor web Apache}
Apache (http://httpd.apache.org/) es un servidor web Open Source y multiplataforma, utilizado para realizar servicio a paginas web estáticas o dinámicas.
Su elección de debe a que es el servidor web más conocido, además de tratarse de un software Open Source. Se adapta perfectamente a nuestras necesidades en el trabajo de fin de grado.


\subsection{Drupal}
Drupal (https://www.drupal.org/) es un gestor de contenido Open Source que posee una gran comunidad, es combinable con MySQL. Posee una gran comunidad de módulos sobre los que apoyarse.
Su sencillez de combinación con MySQL y el hecho de que se trata de un software Open Source, nos han hecho decantarnos por este CMS. La gran cantidad de módulos también han contribuido a su uso.
La versión elegida ha sido Drupal 7 dado que esta tiene una madurez mucho mayor que Dupal 8, sobre todo para la elaboración de una pequeña web como la nuestra, la falta de tiempo para aventurarse y la comunidad detrás han sido claves.

\subsection{Sublime Text}

Sublime Text (https://www.sublimetext.com/) nos permiten editar código fuente de un programa, ayudando en la simplificación de la escritura y resaltando la sintaxis haciendo mas sencillo escribir el código es un editor de texto gratuito, que no libre que nos permite trabajar con una gran variedad de idiomas. En concreto nos permite trabajar con PHP, que ha sido el lenguaje en el que hemos desarrollado gran parte del proyecto.

\subsection{Zotero}

Zotero (https://www.zotero.org/) es una herramienta de gestión de información que  nos ayuda a gestionar las referencias bibliográficas. Obetenemos las referencias que deseamos utilizando la extensión para el navegador Chrome y lo exportamos en formato BibTex a Latex.

\subsection{TexMaker}

TexMaker (http://www.xm1math.net/texmaker) es una moderna plataforma que integra as diferentes herramientas que se ncesitan para desarrollar documentos con LaTeX. Su uso viene motivada por tratarse de una de las mejores herramientes para LaTeX y que tiene una licencia GPL.
