\capitulo{2}{Objetivos del proyecto}

Actualmente podemos encontrar una gran cantidad de información sobre cualquier cosa que podamos imaginar, esta información muchas veces aparece reflejada en estadísticas. El fútbol no es una excepción, en los análisis de los partidos se pueden detectar que algunos equipos tienden a perder cuando, por ejemplo tienen una menor posesión. Es este escenario donde se intenta plantear este proyecto, dado que pudiendo observar las tendencias de los equipos en sus partidos y sus rachas, el algoritmo puede entender las tendencias de cada equipo.

\section{Cálculo del resultado final}

La red neuronal debe ser capaz de pronosticar el resultado de un partido de fútbol basándose en la experiencia adquirida de los pasados partidos y de las rachas calculadas. Dado que se trata de una tarea de Machine Learning con aprendizaje supervisado es necesaria una clase, que será 1-X-2.

\section{Interfaz intuitiva para el usuario}

Esta información la hacemos llegar al usuario a través de un entorno sencillo donde el usuario no solo dispone de la información del resultado, sino que también le proporcionamos información de las diferentes cuotas de algunas casas de apuestas, para que de esta manera, pueda elegir en cual apostar. La manera de mostrarle los datos al usuario es sencilla, <<1>> si se trata de la victoria del equipo local, <<X>> si el partido va a acabar empate y <<2>> si la victoria va a caer del lado visitante, se ha decidido mostrar de esta manera ya que en España estamos acostumbrados a esta simbología gracias a la Quiniela.

\section{Automatización del algoritmo}
La intervención humana deber ser mínima y por este motivo es necesario automatizar el proyecto. La necesidad de automatización viene de que no puede haber una persona pendiente de cuándo se deben ejecutar cada uno de los algoritmos.
Es por ello que será necesaria la creación de un servicio de Linux que sepa cuando debe ejecutar cada uno de los algoritmos.
