\apendice{Especificación de diseño}

\section{Introducción}

En este apéndice vamos a tratar de todo aquello relacionado a la organización del proyecto.

\section{Diseño de datos}

A lo largo de este proyecto la cantidad de datos con la que hemos trabajado es muy grande, dada la necesidad de datos de una red neuronal para un buen pronóstico. En la sección vamos a ver como hemos almacenado estos datos y que uso hemos hecho de estos.

\subsection{Almacenamiento de datos}

Los datos han sido almacenados en el sistema de gestión de bases de datos relacionales MySQL, estala hemos administrado utilizando PHPMyAdmin. Dada la gran cantidad de datos y su diversidad ha sido necesario trabajar con diferentes tablas. A continuación se van a explicar cada una de las tablas utilizadas:

\begin{itemize}
\item \textbf{equipos} En esta tabla encontramos los 20 equipos de primera división, se han dispuesto alfabéticamente y esta tabla posee tres campos, entre ellos está el nombre, este campo es muy importante a la hora de realizar el scraing, ya que es así como identifica la web a cada equipo, el tercer y último campo es <<nombreCompleto>>, utilizado para mostrar el nombre de dicho equipo en la interfaz.

\item \textbf{fecha\_jornada}Esta tabla contiene cada partido de cada jornada de la liga, con algunos campos como la fecha, la jornada o los equipos contendientes.

\item \textbf{clasificacion\_jornada} Aquí podemos observar la posición en la clasificación de cada uno de los equipos en cada jornada, así comolas últimas rachas de los equipos. Esta tabla resulta fundamental a la hora de insertar datos en la red neuronal.

\item \textbf{partidos} Mediante las técnicas de scraping extraemos datos de cada partido, estos datos son almacenados aquí y contiene todas las estadísticas de ambos equipos. Los datos son utilizados para la entrada en la red neuronal.

\item \textbf{pronosticos} Esta tabla contiene la salida de datos de la red neuronal y el resultado final obtenido por esta para cada uno de lo partidos.

\item \textbf{apuestas} Organiza todas las cuotas de las casas de apuestas para cada partido, registrando todos los identificadores.

\item \textbf{cuotas} Posee las cuotas y sus valores dependiendo de la victoria local, empate o victoria visitante.

\item \textbf{balance\_general} Almacena el balance de cada casa de apuestas en cada jornada, esta tabla se utiliza posteriormente para mostrarla en la interfaz.
\end{itemize}
\section{Diseño procedimental}
En esta sección se va a poder observar el funcionamiento del proyecto.

\subsection{Diagrama de flujo}
En esta imagen \ref{fig:DiagFluj} podemos ver la ejecución del programa y de su dependencia del día en el que nos encontremos. Los campos \textit{fecha_antes} y \textit{fecha_despues} son el día previo y posterior a una jornada. Usualmente es trata del viernes a las 00:00 y  de Martes a las 00:00, esto es así por que el primer partido de la jornada suele disputarse el viernes a las 20:45 y el último el lunes a las 20:45.
\begin{figure}
\centering
\includegraphics[width=.9\textwidth]{/img/Diagrama_flujo}
\caption{Diagrama de flujo}
\label{fig:DiagFluj}
\end{figure}

\section{Diseño arquitectónico}


