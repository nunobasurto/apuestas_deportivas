\capitulo{3}{Conceptos teóricos}

En aquellos proyectos que necesiten para su comprensión y desarrollo de unos conceptos teóricos de una determinada materia o de un determinado dominio de conocimiento, debe existir un apartado que sintetice dichos conceptos.


\section{Scraping}
Técnica utilizada para simular  la navegación de un humano en Internet con el fin de extraer información del sitio web. Los datos pueden ser almacenados y analizados en una base de datos central o en otros lugares de almacenamiento.  Son diversas las técnicas que se pueden utilizar, en concreto se ha empleado\cite{_web_2016}

\section{Machine Learning}
Aquel proceso que le da a las computadoras la habilidad de aprender sin ser explícitamente programadas. Hay una segunda definición mucho más clara: "Se dice que un programa de computación aprende de la experiencia E con respecto a una tarea T y alguna medida de rendimiento P, si el rendimiento en T, medido por P, mejora con la experiencia E". 
El Machine Leraning se divide en dos áreas principales, el aprendizaje supervisado y el no supervisado, mientras que el primero requiere la intervención humana, el segundo no.\cite{_machine_learning_2014}

\section{Minería de datos} Es el proceso por el cual es posible detectar información procesable de conjuntos de datos, para después transformarla en una estructura comprensible y poder así utilizarla más tarde.
Tiende a confundirse con el Machine Learning, pero su principal diferencia es que mientras este último se usa para reproducir patrones conocidos y hacer predicciones basadas en patrones, la minería de datos descubre patrones desconocidos.
\section{Neuronas artificiales}
Similares a la idea de neuronas biológicas, forman parte de redes neuronales artificiales. Reciben una serie de entradas y dan una salida, la cual se ve condicionada por tres funciones: función de propagación, función de activación y función de transferencia.\cite{_servidor_2016}

\section{Redes neuronales artificiales}
Imitan el funcionamiento de las redes neuronales biológicas, donde un conjunto de neuronas artificiales trabajan unidas, a fin de resolver problemas relacionados con el reconocimiento de formas o con la predicción. Se parte de un conjunto de datos de entrada significativo para conseguir que la red aprenda las propiedades deseadas.

\section{Aprendizaje Supervisado}
Se dispone de un conjunto de ejemplos de el cual se conoce la respuesta, por lo que el objetivo es marcar una regla o correspondencia de manera que seaposible aproximar la respuesta para todos los objetos que se presenten. La salida de la función puede ser un valor numérico o una clase.\cite{contributors_phpmyadmin_????}
El uso más extendido del aprendizaje supervisado consiste en hacer predicciones a futuro basadas en comportamientos o características ya vistas en los datos que se contienen.\cite{_machine_learning_2014}

\section{Backpropagation}
También denominada propagación hacia atrás de errores, es un tipo de algoritmo con aprendizaje supervisado utilizado para el entrenamiento de las redes neuronales artificiales. Una vez aplicado un patrón, este se propaga a través del resto de capas hasta generar una salida, la cual se compara con la salida establecida como objetivo ya que se trata de aprendizaje supervisado. Se obtiene un error que se propaga hacia atrás cambiando los pesos, acercando así el algoritmo a un mejor resultado.
A medida que entrena la red neuronal, las capas intermedias aprenden a organizarse para así reconocer diferentes características del espacio de entrada.\cite{mazur_step_2015}

\section{Preprocesamiento}
Los datos reales conducen en numerosas ocasiones a la extracción de patrones y reglas poco útiles, lo cual puede deberse al ruido de los datos o a su inconsistencia. La finalidad del preprocesamiento es la creación de información homogénea.

\section{Demonios}
También llamados servicios, son un tipo especial de procesos que se ejecutan en segundo plano.(Wkipeda)
El sistema los inicia en el arranque, ejecutándose en este caso una tarea planificada y comprobando si la fecha del sistema coincide con las fechas de la jornada. Para la ejecución del demonio se emplea un gestor de servicios, el cron. El demonio también es posible ejecutarlo a mano ya que es posible que en un momento determinado no se encuentre el servidor iniciado y no haya sido posible la ejecución del demonio.