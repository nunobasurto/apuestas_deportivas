\capitulo{7}{Conclusiones y Líneas de trabajo futuras}

\section{Conclusiones}

El proyecto que se ha llevado a cabo me parece muy ambicioso, el número de técnicas que abarca es bastante grande, gracias a ello he podido aprender bastante.
Me hubiese gustado disponer de más tiempo para realizar estudios con una mayor exactitud ello hubiese permitido mejores resultados en la red neuronal. Pese a ello me encuentro muy satisfecho del trabajo llevado a cabo, ya que teniendo en cuenta el tiempo disponible me parece que ha sido un buen proyecto.

Los conocimientos adquiridos durante la realización de la carrera universitaria, me han ayudado de gran manera a llevar a cabo el proyecto. Las principales asignaturas sobre las que me he apoyado son:

\begin{itemize}
\item \textbf{Computación neuronal y evolutiva:} Sin duda la asignatura con la que guarda más relación ya que todos los conceptos de redes neuronales, así como el algoritmo de backpropagation han sido extraídos de aquí.

\item \textbf{Sistemas operativos: } La navegación por Ubuntu mediante líneas de comandos, así como la creación de demonios son conocimientos vistos en esta asignatura.

\item \textbf{Bases de datos: }El funcionamiento de la base de datos y las diferentes consultas llevadas a cabo, no hubiesen sido posible realizarlas sin los conocimientos tratados en esta asignatura. Aunque vimos PostgreSQL y aquí se ha trabajado con MySQL las similitudes son grandes.

\item \textbf{Gestión de proyectos: } Sin duda otra de las asignaturas fundamentales en la planificación y gestión del proyecto, las metodologías ágiles han sido necesarias.

\end{itemize}

En cuanto a los conocimientos de la carrera que se han echado un poco de menos, una asignatura en la que aprender programación web sobre la cual apoyarnos para el desarrollo de los scripts en PHP o el web scraping.
\section{Líneas de trabajo futuras}
La escalabilidad que posee este proyecto es muy grande, sin duda puede continuarse en un entorno de apuestas deportivas incluyendo otros deportes, o es más en el propio fútbol cabe la posibilidad de pronosticar una mayor cantidad de tipos de apuestas como por ejemplo los goles a favor, los coners etc...

\subsection{Incremento de la información}
En lo primero que trabajaría sería en ofrecer al algoritmo una mayor cantidad de información con la que aprender, como los jugadores disponibles que tiene o los partidos de descanso que ha tenido el equipo si se ha disputado una competición entre semana. 

Sin duda hay más factores a valorar que no han podido tenerse en cuenta, ya sea por falta de tiempo o por falta de recursos para recopilar esos datos.

\subsection{Ampliación del entorno de actuación}
El trabajo es ampliable a otros deportes como el baloncesto o el balonmano, sin llevar a cabo una gran modificación del mismo. Sin duda hubiese sido atractivo abarcar un mayor número de ligas europeas así como las competiciones europeas.

Fuera del fútbol, hubiese sido atractivo tratar con la NBA, sin duda la liga de baloncesto con mayor número de seguidores en el mundo, donde la gran cantidad de partidos que se disputan nos darían muchos datos con los que trabajar.

\subsection{Pronóstico en tiempo real}
Sin duda se trata de una de las tareas más complejas y a la vez más atractivas sobre las que trabajar en este proyecto. Que el usuario pueda acceder en tiempo real durante la disputa de partido a los pronósticos, supone un cambio en el proyecto pero seguiría los objetivos marcados por el mismo.