\capitulo{7}{Conclusiones y Líneas de trabajo futuras}

Todo proyecto debe incluir las conclusiones que se derivan de su desarrollo. Éstas pueden ser de diferente índole, dependiendo de la tipología del proyecto, pero normalmente van a estar presentes un conjunto de conclusiones relacionadas con los resultados del proyecto y un conjunto de conclusiones técnicas. 
Además, resulta muy útil realizar un informe crítico indicando cómo se puede mejorar el proyecto, o cómo se puede continuar trabajando en la línea del proyecto realizado. 

Los objetivos que se tenían al comienzo del proyecto se han cumplido ya que hemos logrado una interfaz intuitiva, la automatizacón del algoritmo y el princiapl objetivo el pronóstico de resultados de fútbol.

\section{Líneas de trabajo futuras}
La escalabilidad que posee este proyecto es muy grande sin duda, puede continuarse en un entorno de apuestas deportivas incluyendo otros deportes como el tenis o el baloncestos, o es más en el propio fútbol cabe la posibilidad de pronosticar una mayor cantidad de tipos de apuestas como por ejemplo los goles a favor, los coners etc...

\subsection{Incremento de la información}
En lo primero que trabajaría serían en ofrecer al algoritmo una mayor cantidad de información con la que aprender, como los jugadores disponibles que tiene o los partidos de descanso que ha tenido el equipo si se ha disputado una competición entre semana. 

Sin duda hay más factores a valorar que no han podido tenerse en cuenta, ya sea por falta de tiempo o por falta de recursos para recopilar esos datos.

\subsection{Ajuste de la automatización}
Un aspecto a tener en cuenta sería que el algoritmo se automatizara en el transcurso de la jornada, ahora mismo la información se actualiza en la previa o posteriormente a la jornada. 

\subsection{Pronóstico en tiempo real}
Sin duda se trata de una de las tareas más complejas y a la vez más atractivas sobre las que trabajar en este proyecto. Que el usuario pueda acceder en tiempo real durante la disputa de partido a los pronósticos, supone un cambio en el proyecto pero seguiría los objetivos marcados por el mismo.